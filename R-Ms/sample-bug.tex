\documentclass[12pt]{article}
\usepackage[breaklinks=true]{hyperref}
\usepackage{color}
\usepackage{amsmath,amssymb,amsthm}
\usepackage{natbib}
\usepackage{array}
\usepackage{booktabs, multicol, multirow}
\usepackage[nohead]{geometry}
\usepackage[singlespacing]{setspace}
\usepackage[bottom]{footmisc}
\usepackage{floatrow}
\usepackage{float,graphicx}
\usepackage{caption}
\usepackage{indentfirst}


\newtheorem{theorem}{Theorem}[section]
\newtheorem{lemma}[theorem]{Lemma}
\newtheorem{assumption}{Assumption}

\newcommand{\beq}{\begin{equation}}
\newcommand{\eeq}{\end{equation}}


\newcommand{\todo}[1]{{\color{red}{TO DO: \sc #1}}}

\newcommand{\reals}{\mathbb{R}}
\newcommand{\integers}{\mathbb{Z}}
\newcommand{\naturals}{\mathbb{N}}
\newcommand{\rationals}{\mathbb{Q}}

\newcommand{\ind}{\mathbb{I}} % Indicator function
\newcommand{\pr}{\mathbb{P}} % Generic probability
\newcommand{\ex}{\mathbb{E}} % Generic expectation
\newcommand{\var}{\textrm{Var}}
\newcommand{\cov}{\textrm{Cov}}

\newcommand{\normal}{N} % for normal distribution (can probably skip this)
\newcommand{\eps}{\varepsilon}
\newcommand\independent{\protect\mathpalette{\protect\independenT}{\perp}}
\def\independenT#1#2{\mathrel{\rlap{$#1#2$}\mkern2mu{#1#2}}}
\newcommand{\argmax}{\textrm{argmax}}
\newcommand{\argmin}{\textrm{argmin}}
\renewcommand{\baselinestretch}{1.5}

\title{Simple Random Sampling: Not So Simple}
\author{Kellie Ottoboni and Philip~B. Stark}
\date{Draft \today}
\begin{document}
\maketitle


\begin{abstract}
R (Version 3.5.0) generates random integers between $1$ and $m$
by multiplying random floats by $m$, taking the floor, and adding $1$ to the result.
It is well known that quantization effects inherent in this approach result in a 
non-uniform distribution on $\{ 1, \ldots, m\}$.
The difference, which depends on $m$, can be substantial.
Because the \texttt{sample} function in R relies on generating random integers,
random sampling in R is also biased.
There is an easy fix, namely, construct random integers directly from random bits, rather than
multiplying a random float by $m$.
That is the strategy taken in Python's \texttt{numpy.random.randint()} function, among
others.
\end{abstract}


%\newpage

%\section{Introduction}
A textbook way to generate a random integer on 
$\{1, \dots, m\}$ is to start with $X \sim U[0,1)$ and define $Y \equiv 1 + \lfloor mX \rfloor$. 
If $X$ is truly uniform on $[0,1)$, $Y$ is then uniform on $\{1, \dots, m\}$.
However, if $X$ has a discrete distribution derived by scaling a pseudorandom binary integer, 
the resulting distribution is not uniformly distributed on 
$\{1, \ldots, m \}$ even if the underlying pseudorandom number generator 
(PRNG) is perfect (unless $m$ is a power of 2):

\begin{theorem}[\citet{knuth_art_1997}] % p.133
Suppose $X$ is uniformly distributed on $w$-bit binary numbers, and
let $Y_m \equiv 1 + \lfloor mX \rfloor$.
Let $p_+(m) = \max_{1 \le k \le m} \Pr\{Y_m = k\}$ and $p_-(m) = \min_{1 \le k \le m} \Pr\{Y_m = k\}$.
There exists $m < 2^w$ such that, to first order, 
$p_+(m)/p_-(m) = 1 + m2^{-w+1}$.
\end{theorem}

The algorithm that R (Version 3.5.0) \citep{R_2018} uses to generate uniform random integers
has this issue (albeit in a slightly more complicated form, because, depending on $m$,
R uses pseudorandom binary integers of different lengths). 
Because \texttt{sample} relies on random integers, it inherits the problem.

R uses \texttt{unif\_rand} to generate pseudorandom numbers with word size at most $w=32$.
To generate integers with a larger word size, R combines two $w$-bit integers to obtain an integer with 50 to 53 bits (depending the chosen PRNG). 

A better way to generate random elements of $\{1, \dots, m\}$ is to use pseudorandom bits directly. 
The integer $m$ can be represented with $\mu = \lceil \log_2(m) \rceil$ bits. 
To generate a pseudorandom integer between $1$ and $m$, first generate $\mu$ pseudorandom bits (for instance, by taking the most significant $\mu$ bits from the PRNG output).  
If that binary number $M$ is larger than $m-1$, discard it.
Repeat until the $\mu$ bits represent an integer $M$ that is less than $m$. 
Return $M+1$.\footnote{%
   See \citet{knuth_art_1997} p.114.
}
This procedure might discard almost half the draws if $m$ is slightly larger than a power of $2$,
but if the input bits were IID Bernoulli(1/2), the resulting integers will be uniformly distributed.
This is how the Python function \texttt{numpy.random.randint()} (Version 1.14) generates pseudorandom integers.\footnote{%
However, Python's built-in \texttt{random.choice()} (Versions 2.7 through 3.6) does something else biased: it finds the closest integer to $mX$, where $X$ is a binary fraction between 0 and 1.
}

The R \texttt{sample} function has a branch in its logic depending on the number of elements
in the population to be sampled. 
It uses \texttt{ru} when $m >= 2^{31}$ and \texttt{rand\_unif} when $m < 2^{31}$.\footnote{
A different function, \texttt{sample2}, is called when $m > 10^7$ and $k < m/2$.
It uses the same flawed method of generating pseudorandom integers.
}
The nonuniformity of selection probabilities is largest when $m$ is just below $2^{31}$. 
In that case, \texttt{sample} calls \texttt{unif\_rand}, which gives outputs with word size $w=32$. 
The maximum ratio of selection probabilities approaches $2$ as $m$ increases to the cutoff $2^{31}$, or about 2 billion. 
Even if $m$ is close to 1 million, the ratio is about $1.0004$.

When $m \ge 2^{31}$, R calls \texttt{ru()}  to produce a pseudorandom number with word length at least $w=50$ bits and at most $w=53$ bits (depending the chosen PRNG). 
In that branch of the logic, ratio of selection probabilities only becomes large (on the order of $1+10^{-3}$) for large population sizes, say $m > 2^{40} \approx 10^{12}$. 

\begin{table}[h]
\caption{Maximum ratio of selection probabilities for different population sizes $m$. 
R uses bases its random integers on random binary integers with different numbers of bits (wordlengths) depending on $m$}
\begin{center}
\begin{tabular}{|c|c|r|}

\hline
Population size ($m$) & Word length ($w$) & Maximum ratio of selection probabilities\\
\hline 
$10^6$ & 32 & 1.0004 \\
$10^9$ & 32 & 1.466 \\
 $2^{31}-\epsilon$ & 32 & 2 \\
$2^{31}+\epsilon$ & 53 & $1 + 2.3 \times 10^{-7}$ \\
$10^{12}$ & 53 & $1.0001$ \\
$10^{15}$ & 53 & $1.11$ \\
\hline

\end{tabular}
\end{center}
\label{tab}
\end{table}%

We recommend that the R developers replace the current algorithm for generating pseudo-random integers with the masking algorithm.

\bibliographystyle{plainnat}
\bibliography{refs}

\end{document}
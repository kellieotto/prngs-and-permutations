\documentclass[12pt]{article}
%\usepackage[breaklinks=true]{hyperref}
\usepackage{color}
\usepackage{amsmath,amssymb,amsthm}
\usepackage{natbib}
\usepackage{array}
\usepackage{booktabs, multicol, multirow}
\usepackage[nohead]{geometry}
\usepackage[singlespacing]{setspace}
\usepackage[bottom]{footmisc}
\usepackage{floatrow}
\usepackage{float,graphicx}
\usepackage{caption}
\usepackage{indentfirst}
\usepackage{lscape}
\usepackage{floatrow}
\usepackage{epsfig}
\usepackage[usenames,dvipsnames,svgnames,table]{xcolor}
\usepackage[colorlinks=true,
            urlcolor=RawSienna,
            linkcolor=RawSienna,
            citecolor=NavyBlue]{hyperref}


\newtheorem{theorem}{Theorem}[section]
\newtheorem{lemma}[theorem]{Lemma}
\newtheorem{assumption}{Assumption}

\newcommand{\beq}{\begin{equation}}
\newcommand{\eeq}{\end{equation}}


\newcommand{\todo}[1]{{\color{red}{TO DO: \sc #1}}}

\newcommand{\reals}{\mathbb{R}}
\newcommand{\integers}{\mathbb{Z}}
\newcommand{\naturals}{\mathbb{N}}
\newcommand{\rationals}{\mathbb{Q}}

\newcommand{\ind}{\mathbb{I}} % Indicator function
\newcommand{\pr}{\mathbb{P}} % Generic probability
\newcommand{\ex}{\mathbb{E}} % Generic expectation
\newcommand{\var}{\textrm{Var}}
\newcommand{\cov}{\textrm{Cov}}

\newcommand{\normal}{N} % for normal distribution (can probably skip this)
\newcommand{\eps}{\varepsilon}
\newcommand\independent{\protect\mathpalette{\protect\independenT}{\perp}}
\def\independenT#1#2{\mathrel{\rlap{$#1#2$}\mkern2mu{#1#2}}}
\newcommand{\argmax}{\textrm{argmax}}
\newcommand{\argmin}{\textrm{argmin}}
\renewcommand{\baselinestretch}{1.5}

\begin{document}

\title{Simple Random Sampling: Not Simple}
\author{Kellie Ottoboni
\and
Ron L. Rivest
\and
Philip B.~Stark 
}

\date{draft \today}




\maketitle

\begin{abstract}
\small
A simple random sample (SRS) of size $k$ from a population of size $n$ is a sample drawn 
at random in such a way that every subset of $k$ of the $n$ items is equally likely to be selected. 
The theory of inference from SRSs is fundamental in statistics;
many statistical techniques and formulae assume that the data are an SRS.
True SRSs are rare; in practice, people tend to draw samples by using pseudo-random number generators 
(PRNGs) and algorithms that map a set of pseudo-random numbers into a subset of the population. 
Most statisticians take for granted that the software they use ``does the right thing,''
producing samples that can be treated as if they are SRSs.
In fact, the PRNG algorithm and the algorithm for drawing samples using the PRNG matter
enormously.
Using basic counting principles, we show that some widely used methods cannot generate all subsets of size $k$.
In simulations, we demonstrate that the subsets that they do generate do not have equal frequencies, which
introduces bias and makes uncertainty calculations meaningless.
We compare the ``randomness'' and computational efficiency of commonly-used PRNGs to a PRNG 
based on the SHA-256 hash function, which avoids these pitfalls because its state space is countably infinite.
We propose several best practices for researchers using PRNGs, including the wide adoption of hash function based PRNGs.
\end{abstract}

\section{Introduction}
Random sampling is one of the most fundamental tools in Statistics.
It is used to conduct surveys, including opinion surveys, population surveys like the census, and litigation; 
to run medical, agricultural, and marketing experiments; 
quality control in industry and auditing in finance and elections;
and countless other purposes.
Simple random sampling refers to drawing $k \leq n$ items from a population of $n$ items,
in such a way that each of ${n \choose k}$ subsets of size $k$ is equally likely.
Many standard statistical methods assume that the sample is drawn this way, 
or allocated between treatment and control groups this way
(e.g. $k$ of $n$ subjects are assigned to treatment, and the remaining $n-k$ to control).

\begin{itemize}
\item We examine methods for drawing pseudo-random simple random samples. 
We include a discussion of pseudo-random number generators (PRNGs) and 
of algorithms used to select samples using PRNGs.
Among other things, we find bounds on the number of samples that can be generated 
using a variate of PRNGs, for a number of sampling algorithms. 
We also consider how that affects the bias and uncertainty of estimates based on pseudo-random
samples rather than on actual simple random samples.
\item PRNGs considered include linear congruential generators (LCGs, including RANDU)
and the Mersenne Twister. We discuss using cryptographic hash functions to generate PRNs.
\item We conclude with recommendations for best practices using PRNGs to generate random samples.
\end{itemize}



\section{Background}
\subsection{Definition of ``random'' numbers}
Most computers lack the hardware needed to generate truly random numbers. 
Instead, they use algorithms called pseudo-random number generators (PRNGs) to generate
deterministic sequences from an initial ``seed,'' which generally can be set by the user,
for instance to an externally generated random value.
Each time a number is generated, the PRNG's ``state'' changes.

Depending on the quality of the PRNG, the sequences behave more or less like sequences of random numbers.
How does one gauge ``how random'' sequences from a PRNG are?
A PRNG yields sequences of random numbers on the interval $[0, 1]$ or over the binary set $\{0, 1\}$.
The sequences output by such PRNGs should be statistically indistinguishable from IID $U(0,1)$ sequences or
IID Bernoulli$(1/2)$ sequences, respectively.
The tests for the bit sequence case applies to the $U(0,1)$ case as well, because there is a one-to-one relationship
between sequences of bits and integers.
Namely, every sequence of $k$ bits corresponds to an integer on $0, 1, \dots, 2^k-1$, and scaling by $2^k$
yields a unique value on $[0, 1)$. (cite L'Ecuyer, Simard)

There are a multitude of ways to test the hypothesis that sequences from a PRNG are indistinguishable 
from random sequences.
\todo{list types of tests + citations: L'Ecuyer Simard TestU01 (2007),  Knuth (1969), Marsaglia DIEHARD tests (1968)}



\subsection{RANDU and LCGs}
\subsection{Stata software}
\section{Instances of the problem}
\subsection{Random integer generation in R}
\subsection{Pigeonhole arguments}
\subsection{Simulations showing bias}
\section{Hash-function based RNGs}
\section{Discussion}
\subsection{Best Practices}
\section{Conclusions}




\end{document}
